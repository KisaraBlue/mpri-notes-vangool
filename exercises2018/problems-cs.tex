\documentclass[11pt]{book}

\usepackage{amsmath,amssymb,amsthm}
\usepackage{array}
\usepackage[utf8]{inputenc}
\usepackage{tikz}
\usetikzlibrary{shapes,backgrounds,matrix,arrows,automata,positioning}
\usepackage{tikz-cd}
\usepackage{scalefnt}

%% TIKZ AUTOMATA MACROS

\newcommand{\finalarr}[1]{
  \node (FINAL#1) [right=0.5cm of #1] {};
  \path (#1) edge (FINAL#1);
  }

\newcommand{\initarr}[1]{
  \node (INIT#1) [left=0.5cm of #1] {};
  \path (INIT#1) edge (#1);
  }



\usepackage{enumitem}

\usepackage{verbatim}

\usepackage[hidelinks]{hyperref}

\usetikzlibrary{matrix,arrows}

%set margins
\usepackage[top=3cm,bottom=3cm,left=3cm,right=3cm]{geometry}
\parindent=0pt

%set linespread
\usepackage{setspace}
\linespread{1}


%theorem styles
\theoremstyle{plain}
\newtheorem{theorem}{Theorem}
\newtheorem*{theoremstar}{Theorem}
\newtheorem{proposition}[theorem]{Proposition}
\newtheorem{lemma}[theorem]{Lemma}
\newtheorem{corollary}[theorem]{Corollary}

\theoremstyle{definition}
\newtheorem{definition}[theorem]{Definition}
\newtheorem{example}[theorem]{Example}
\newtheorem{remark}[theorem]{Remark}
\newtheorem*{remarknonum}{Remark}
\newtheorem*{notation}{Notation}
\newtheorem{exercise}[theorem]{Exercise}
\newtheorem*{zornslemma}{Zorn's Lemma}
\newtheorem*{alexanderlemma}{Alexander Subbasis Lemma}

\renewcommand{\thetheorem}{\arabic{theorem}}
%\renewcommand{\thesection}{\arabic{section}}
\renewcommand{\theenumi}{\alph{enumi}}

\newcommand{\exercises}{\subsection*{Exercises for Section~\thesection}}

\numberwithin{theorem}{chapter}
%\numberwithin{exercise}{chapter}
%arrows
\newcommand{\onto}{\twoheadrightarrow}
\newcommand{\twoheaduparrow}{{\rotatebox[origin=c]{90}{$\twoheadrightarrow$}}}
\newcommand\twoheaddownarrow{\mathrel{\rotatebox[origin=c]{-90}{$\twoheadrightarrow$}}}
\newcommand{\To}{\Rightarrow}
\newcommand{\gen}[1]{\langle #1 \rangle}
\newcommand{\into}{\hookrightarrow}

%greek letters
\renewcommand{\theta}{\vartheta}
\renewcommand{\phi}{\varphi}


\newcommand{\cB}{\mathcal{B}}
\newcommand{\cC}{\mathcal{C}}
\newcommand{\cD}{\mathcal{D}}
\newcommand{\cF}{\mathcal{F}}%closed subsets
\newcommand{\cK}{\mathcal{K}}%compact saturated subsets
\newcommand{\cL}{\mathcal{L}}
\newcommand{\cN}{\mathcal{N}}
\newcommand{\cP}{\mathcal{P}}%powerset
\newcommand{\cS}{\mathcal{S}}
\newcommand{\cT}{\mathcal{T}}

\newcommand{\Top}{\mathit{Top}}

%categories
\newcommand{\cat}[1]{\ensuremath{\mathbf{#1}}}
\newcommand{\BA}{\cat{BA}}
\newcommand{\Pos}{\cat{Pos}}
\newcommand{\DL}{\cat{DL}}
\newcommand{\Set}{\cat{Set}}
\newcommand{\POS}{\cat{Pos}}
\newcommand{\PreOS}{\cat{PreOS}}
\newcommand{\dcpo}{\cat{dcpo}}
\newcommand{\TOP}{\cat{Top}}
\newcommand{\SOB}{\cat{Sober}}
\newcommand{\CONT}{\cat{Domain}}
\newcommand{\Stone}{\cat{Stone}}
\newcommand{\BStone}{\cat{BStone}}
\newcommand{\Frame}{\cat{Frame}}
\newcommand{\SpFrame}{\cat{SpFrame}}
\newcommand{\ArithFr}{\cat{ArithFr}}
\newcommand{\Priest}{\cat{Priestley}}

\newcommand{\R}{\mathbb{R}}
\newcommand{\utwo}{\underline{\bf 2}}
\newcommand{\btwo}{{\bf 2}}


\newcommand{\cA}{\mathcal{A}}
\newcommand{\cR}{\mathcal{R}}

\newcommand{\Pt}{\mathrm{Pt}}
\newcommand{\St}{\mathrm{St}}

\newcommand{\Up}{\mathcal{U}} %up-sets
\newcommand{\Down}{\mathcal{D}} %down-sets
\newcommand{\ClD}{\mathcal{KD}} %clopen down-sets
\newcommand{\Cl}{\mathcal{K}} % clopens
\newcommand{\CO}{\mathrm{CO}}% Compact-Opens
\newcommand{\Po}{\mathcal{P}} %power set
\newcommand{\op}{\mathrm{op}} %opposite
\newcommand{\J}{\mathcal{J}} %join-irreducibles
\newcommand{\Comp}{\it{Comp}} %compact elements
\newcommand{\M}{\mathcal{M}} %meet-irreducibles
\newcommand{\Filt}{\mathrm{Filt}} % Filters
\newcommand{\Idl}{\mathrm{Idl}} % Ideals
\newcommand{\PrFilt}{\mathrm{PrFilt}} % Prime Filters
\newcommand{\PrIdl}{\mathrm{PrIdl}} % Prime ideals
\newcommand{\CompPrFilt}{\mathrm{CompPrFilt}} % Completely Prime Filters
\DeclareMathOperator{\ob}{\mathrm{ob}}
\DeclareMathOperator{\mor}{\mathrm{mor}}
\DeclareMathOperator{\cod}{\mathrm{cod}}
\DeclareMathOperator{\dom}{\mathrm{dom}}

%logic symbols
\newcommand{\Prop}{\mathsf{Prop}} %basic propositions
\renewcommand{\Form}{\mathrm{Form}} %formula algebra
\newcommand{\CPL}{\mathsf{CPL}}
\newcommand{\liff}{\leftrightarrow}

\newcommand{\Hom}{\mathrm{Hom}}
\newcommand{\id}{\mathrm{id}}
%notes at end of chapter
\usepackage{endnotes,etoolbox}
% end note marker = superscripted number in brackets
%\renewcommand\makeenmark{\textsuperscript{[\theenmark]}}

% in the endnotes, we change it without `\textsuperscript`, adding a space
\patchcmd{\theendnotes}
  {\makeatletter}
  {\makeatletter\renewcommand\makeenmark{\theenmark. }}
  {}{}

\newcommand{\wayb}{<\!\!\!<}


%% TIKZ MACROS
\newcommand{\labeldist}{.1cm}
\newcommand{\po}[1]{\draw[fill=black] #1 circle (2pt);}
\newcommand{\polab}[3]{\draw[fill=black] #1 circle (2pt); \node[#3=\labeldist] at #1 {#2};}
\newcommand{\li}[2]{\draw [thick] #1 -- #2;}


\newcommand{\easy}{($\star$) }
\newcommand{\medium}{($\star\star$) }
\newcommand{\hard}{($\star\star\star)$ }


% for index

%INDEX PACKAGE
\usepackage{makeidx}
\usepackage[refpage]{nomencl}
\renewcommand{\pagedeclaration}{, }
\renewcommand{\nomname}{Notation}
\makenomenclature
\newcommand{\emphind}[1]{\emph{#1}\index{#1}}



\makeindex


%%% Local Variables:
%%% TeX-master: "mpri-vangool-notes"
%%% End:

\usepackage[final]{pdfpages}
\usepackage{graphicx}

\begin{document}

\thispagestyle{empty}
\begin{center}
{\bf Problem Set: Applications of duality theory to computer science}
\end{center}

{\bf Problem 1.} Let $\Sigma = \{a,b\}$. Consider the regular $\Sigma$-language $L = (ab)^*$. Recall that $B(L)$ is the Boolean algebra generated by $\{u^{-1}Lv^{-1} \ | \ u, v \in \Sigma^*\}$. We saw in the lectures that, in this case, $B(L)$ is generated by the four languages $(ab)^*, a(ba)^*, b(ab)^*,$ and $(ba)^*$. Write $X(L)$ for the set of atoms of $B(L)$.

\begin{enumerate}
	%\item Show that, if $B$ is a finite Boolean subalgebra of $\mathcal{P}(X)$, then the atoms of $B$ are the smallest partition of $X$ consisting of elements of $B$.
	\item Prove that $X(L) = \{x_1,\dots,x_6\}$, where $x_1 := \{\epsilon\}$, $x_2 := (ab)^+$, $x_3 := (ba)^+$, $x_4 := a(ba)^*$, $x_5 := b(ab)^*$ and $x_6 := \Sigma^* \setminus \left(\bigcup_{i=1}^5 x_i\right)$.
	\item Describe the function $\eta \colon \Sigma^* \onto X(L)$ which is dual to the inclusion map $B(L) \into \mathcal{P}(\Sigma^*)$.
	\item Show that, for any $w_1,w_2,w_1',w_2' \in \Sigma^*$, if $\eta(w_1) = \eta(w_1')$, $\eta(w_2) = \eta(w_2')$, then $\eta(w_1w_2) = \eta(w_1'w_2')$. Conclude that $X(L)$ is a finite monoid quotient of $\Sigma^*$.
	\item Is the syntactic monoid $X(L)$ aperiodic? Prove your assertion. (You may either appeal to Sch\"utzenberger's Theorem, or prove it directly.)
\end{enumerate}

{\bf Problem 2.} In this problem, you will prove that there is a bijection between the free pro-aperiodic monoid and the set of elementary equivalence classes of first-order models. Let $\Sigma$ be a finite alphabet and let $T$ be the first-order theory of finite $\Sigma$-words, i.e.,
\[ T = \{\phi \text{ a sentence in } \mathrm{FO}[<,\Sigma] \ | \ W \models \phi \text{ for every finite $\Sigma$-word $W$}\}.\]
For $\phi, \psi$ sentences in $\mathrm{FO}[<,\Sigma]$, write $\phi \equiv_T \psi$ iff $T \vdash \phi \liff \psi$, and denote by $\mathcal{L}_T$ the Lindenbaum algebra of $T$, i.e., the Boolean algebra of $\equiv_T$-equivalence classes of sentences in $\mathrm{FO}[<,\Sigma]$.
\begin{enumerate}
\item Show that the function $i \colon \mathcal{L}_T \to \mathcal{P}(\Sigma)$, which sends $[\phi]_{\equiv_T}$ to $L(\phi)$, is well-defined, and an injective Boolean algebra homomorphism.
\item Explain why Sch\"utzenberger's Theorem, together with item (a), implies that the Boolean algebra $\mathcal{L}_T$ is isomorphic to the Boolean algebra $\mathrm{AP}(\Sigma)$ of $\Sigma$-languages whose syntactic monoid is aperiodic. 
\item For any (possibly infinite) model $W$ of the theory $T$, define a subset $x_W$ of $\mathcal{L}_T$ by 
\[ x_W := \{[\phi]_{\equiv_T} \ | \ W \models \phi\}.\]
Prove that $x_W$ is an ultrafilter of $\mathcal{L}_T$, and that $x_W = x_{W'}$ if, and only if, $W$ and $W'$ are elementarily equivalent.
\item Prove that, for any ultrafilter $x$ of $\mathcal{L}_T$, there exists a model $W$ of $T$ such that $x = x_W$.

 {\it Hint.} Apply the completeness theorem for first-order logic.
\item Use (b)--(d) to prove that the set underlying the free pro-aperiodic monoid over $\Sigma$  is in a bijection with the set of elementary equivalence classes of models of $T$. 

{\it Hint.} Use the fact that the free pro-aperiodic monoid over $\Sigma$ is the dual space of $\mathrm{AP}(\Sigma)$.

{\it Remark.} The bijection constructed in this problem is in fact an isomorphism of topological monoids; see Theorem 10 in [v. Gool \& Steinberg, STACS 2017].
\end{enumerate}
\end{document}