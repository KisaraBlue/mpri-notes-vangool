\documentclass[11pt]{book}

\usepackage{amsmath,amssymb,amsthm}
\usepackage{array}
\usepackage[utf8]{inputenc}
\usepackage{tikz}
\usetikzlibrary{shapes,backgrounds,matrix,arrows,automata,positioning}
\usepackage{tikz-cd}
\usepackage{scalefnt}

%% TIKZ AUTOMATA MACROS

\newcommand{\finalarr}[1]{
  \node (FINAL#1) [right=0.5cm of #1] {};
  \path (#1) edge (FINAL#1);
  }

\newcommand{\initarr}[1]{
  \node (INIT#1) [left=0.5cm of #1] {};
  \path (INIT#1) edge (#1);
  }



\usepackage{enumitem}

\usepackage{verbatim}

\usepackage[hidelinks]{hyperref}

\usetikzlibrary{matrix,arrows}

%set margins
\usepackage[top=3cm,bottom=3cm,left=3cm,right=3cm]{geometry}
\parindent=0pt

%set linespread
\usepackage{setspace}
\linespread{1}


%theorem styles
\theoremstyle{plain}
\newtheorem{theorem}{Theorem}
\newtheorem*{theoremstar}{Theorem}
\newtheorem{proposition}[theorem]{Proposition}
\newtheorem{lemma}[theorem]{Lemma}
\newtheorem{corollary}[theorem]{Corollary}

\theoremstyle{definition}
\newtheorem{definition}[theorem]{Definition}
\newtheorem{example}[theorem]{Example}
\newtheorem{remark}[theorem]{Remark}
\newtheorem*{remarknonum}{Remark}
\newtheorem*{notation}{Notation}
\newtheorem{exercise}[theorem]{Exercise}
\newtheorem*{zornslemma}{Zorn's Lemma}
\newtheorem*{alexanderlemma}{Alexander Subbasis Lemma}

\renewcommand{\thetheorem}{\arabic{theorem}}
%\renewcommand{\thesection}{\arabic{section}}
\renewcommand{\theenumi}{\alph{enumi}}

\newcommand{\exercises}{\subsection*{Exercises for Section~\thesection}}

\numberwithin{theorem}{chapter}
%\numberwithin{exercise}{chapter}
%arrows
\newcommand{\onto}{\twoheadrightarrow}
\newcommand{\twoheaduparrow}{{\rotatebox[origin=c]{90}{$\twoheadrightarrow$}}}
\newcommand\twoheaddownarrow{\mathrel{\rotatebox[origin=c]{-90}{$\twoheadrightarrow$}}}
\newcommand{\To}{\Rightarrow}
\newcommand{\gen}[1]{\langle #1 \rangle}
\newcommand{\into}{\hookrightarrow}

%greek letters
\renewcommand{\theta}{\vartheta}
\renewcommand{\phi}{\varphi}


\newcommand{\cB}{\mathcal{B}}
\newcommand{\cC}{\mathcal{C}}
\newcommand{\cD}{\mathcal{D}}
\newcommand{\cF}{\mathcal{F}}%closed subsets
\newcommand{\cK}{\mathcal{K}}%compact saturated subsets
\newcommand{\cL}{\mathcal{L}}
\newcommand{\cN}{\mathcal{N}}
\newcommand{\cP}{\mathcal{P}}%powerset
\newcommand{\cS}{\mathcal{S}}
\newcommand{\cT}{\mathcal{T}}

\newcommand{\Top}{\mathit{Top}}

%categories
\newcommand{\cat}[1]{\ensuremath{\mathbf{#1}}}
\newcommand{\BA}{\cat{BA}}
\newcommand{\Pos}{\cat{Pos}}
\newcommand{\DL}{\cat{DL}}
\newcommand{\Set}{\cat{Set}}
\newcommand{\POS}{\cat{Pos}}
\newcommand{\PreOS}{\cat{PreOS}}
\newcommand{\dcpo}{\cat{dcpo}}
\newcommand{\TOP}{\cat{Top}}
\newcommand{\SOB}{\cat{Sober}}
\newcommand{\CONT}{\cat{Domain}}
\newcommand{\Stone}{\cat{Stone}}
\newcommand{\BStone}{\cat{BStone}}
\newcommand{\Frame}{\cat{Frame}}
\newcommand{\SpFrame}{\cat{SpFrame}}
\newcommand{\ArithFr}{\cat{ArithFr}}
\newcommand{\Priest}{\cat{Priestley}}

\newcommand{\R}{\mathbb{R}}
\newcommand{\utwo}{\underline{\bf 2}}
\newcommand{\btwo}{{\bf 2}}


\newcommand{\cA}{\mathcal{A}}
\newcommand{\cR}{\mathcal{R}}

\newcommand{\Pt}{\mathrm{Pt}}
\newcommand{\St}{\mathrm{St}}

\newcommand{\Up}{\mathcal{U}} %up-sets
\newcommand{\Down}{\mathcal{D}} %down-sets
\newcommand{\ClD}{\mathcal{KD}} %clopen down-sets
\newcommand{\Cl}{\mathcal{K}} % clopens
\newcommand{\CO}{\mathrm{CO}}% Compact-Opens
\newcommand{\Po}{\mathcal{P}} %power set
\newcommand{\op}{\mathrm{op}} %opposite
\newcommand{\J}{\mathcal{J}} %join-irreducibles
\newcommand{\Comp}{\it{Comp}} %compact elements
\newcommand{\M}{\mathcal{M}} %meet-irreducibles
\newcommand{\Filt}{\mathrm{Filt}} % Filters
\newcommand{\Idl}{\mathrm{Idl}} % Ideals
\newcommand{\PrFilt}{\mathrm{PrFilt}} % Prime Filters
\newcommand{\PrIdl}{\mathrm{PrIdl}} % Prime ideals
\newcommand{\CompPrFilt}{\mathrm{CompPrFilt}} % Completely Prime Filters
\DeclareMathOperator{\ob}{\mathrm{ob}}
\DeclareMathOperator{\mor}{\mathrm{mor}}
\DeclareMathOperator{\cod}{\mathrm{cod}}
\DeclareMathOperator{\dom}{\mathrm{dom}}

%logic symbols
\newcommand{\Prop}{\mathsf{Prop}} %basic propositions
\renewcommand{\Form}{\mathrm{Form}} %formula algebra
\newcommand{\CPL}{\mathsf{CPL}}
\newcommand{\liff}{\leftrightarrow}

\newcommand{\Hom}{\mathrm{Hom}}
\newcommand{\id}{\mathrm{id}}
%notes at end of chapter
\usepackage{endnotes,etoolbox}
% end note marker = superscripted number in brackets
%\renewcommand\makeenmark{\textsuperscript{[\theenmark]}}

% in the endnotes, we change it without `\textsuperscript`, adding a space
\patchcmd{\theendnotes}
  {\makeatletter}
  {\makeatletter\renewcommand\makeenmark{\theenmark. }}
  {}{}

\newcommand{\wayb}{<\!\!\!<}


%% TIKZ MACROS
\newcommand{\labeldist}{.1cm}
\newcommand{\po}[1]{\draw[fill=black] #1 circle (2pt);}
\newcommand{\polab}[3]{\draw[fill=black] #1 circle (2pt); \node[#3=\labeldist] at #1 {#2};}
\newcommand{\li}[2]{\draw [thick] #1 -- #2;}


\newcommand{\easy}{($\star$) }
\newcommand{\medium}{($\star\star$) }
\newcommand{\hard}{($\star\star\star)$ }


% for index

%INDEX PACKAGE
\usepackage{makeidx}
\usepackage[refpage]{nomencl}
\renewcommand{\pagedeclaration}{, }
\renewcommand{\nomname}{Notation}
\makenomenclature
\newcommand{\emphind}[1]{\emph{#1}\index{#1}}



\makeindex


%%% Local Variables:
%%% TeX-master: "mpri-vangool-notes"
%%% End:

\usepackage[final]{pdfpages}
\usepackage{graphicx}

\begin{document}

\thispagestyle{empty}
\begin{center}
{\bf Problem Set: Applications of duality theory to logic}
\end{center}

{\bf Problem 1.} Let $L$ be a finite distributive lattice with dual poset $X := \J(L)$. For any operator $f \colon L \to L$, define $R_f \subseteq X \times X$ by
\[ R_f := \{(x,y) \in X \times X \ | \ x \leq f(y)\}.\]

a. Prove that $R_f$ is order-compatible.\\

For any $R \subseteq X \times X$, define $f \colon \Down(X) \to \Down(X)$ by 
\[ f_R(U) := R^{-1}(U) = \{x \in X \ | \ \exists y \in U \text{ such that } xRy\}.\]

b. Prove that $f_R$ is an operator.\\

c. Let $f \colon L \to L$ be an operator and $\widehat{(-)} \colon L \to \Down(X)$ the lattice isomorphism defined by $\widehat{a} := \{x \in X = \J(L) \ | \ x \leq a\}$. Prove that, for any $a \in L$, $f_{R_f}(\widehat{a}) = \widehat{f(a)}$.\\

d. Let $R \subseteq X \times X$ be an order-compatible relation. Prove that, for any $x \in X$, ${\downarrow} x {R_{f_R}} {\downarrow} y$ if, and only if, $xRy$.\\

e. Prove that an operator $h \colon L \to L$ is a homomorphism if, and only if, $R_h[x]$ has a minimum for every $x \in X$.

{\it Hint.} To prove that a set $S \subseteq X$ has a minimum, it suffices to prove that (i) $S$ is non-empty, and (ii) $S$ is \emph{down-directed}, i.e., whenever $y_1, y_2 \in S$, there exists $y \in S$ with $y \leq y_1$ and $y \leq y_2$ (why?). For the right-to-left direction, begin by showing, using item (c), that, for any $x \in X$ and $a \in L$, $x \leq h(a)$ if, and only if, $\min R_h[x] \leq a$. \\

{\bf Problem 2.} Let $(L,\to)$ be a Heyting algebra, i.e., $L$ is a distributive lattice and $\to$ is a \emph{Heyting implication}, i.e., a binary operation on $L$ such that, for any $a, b, c \in L$, 
\[ a \wedge b \leq c \text{ if, and only if, } a \leq b \to c.\]

{\it {Note}} that a binary operation $\to$ is a Heyting implication on $L$, if, and only if, for every $b \in L$, the pair of maps $f_b \colon L \leftrightarrows L \colon g_b$, defined by 
\[ f_b(a) := a \wedge b, \quad g_b(c) := b \to c,\]
is an adjunction.\\ %Use this observation and earlier exercises on adjunctions to prove the following.\\

a. Show that any lattice admits at most one Heyting implication.\\

b. Show that any lattice which admits a Heyting implication must be distributive. Is this a sufficient condition?\\

c. Show that, for any $a, b_1, b_2 \in L$, $a \to (b_1 \wedge b_2) = (a \to b_1) \wedge (a \to b_2)$, and $a \to \top = \top$.\\

%d. Show that, for any $a_1, a_2, b \in L$, $(a_1 \vee a_2) \Rightarrow b = (a_1 \Rightarrow b) \wedge (a_2 \Rightarrow b)$ and $\bot \Rightarrow a = \top$.\\

d. Prove that, if $(X, \leq)$ is a poset, then the distributive lattice $\Down(X)$ admits a Heyting implication. Conclude that, in particular, any finite distributive lattice admits a Heyting implication.

{\it Hint.} For $U, V \in \Down(X)$, define $U \Rightarrow V := ({\uparrow}(U \cap V^c))^c,$
  and show that this is a Heyting implication. \\

e. (+) Prove that, for any Heyting algebra $(L,\to)$, the embedding $\widehat{(-)} \colon L \to \Down(X_L)$ preserves the Heyting implication.\\


An \emph{Esakia space} is a Priestley space $X$ such that, for any open subset $U \subseteq X$, ${\uparrow} U$ is open. \\

f. (+) Prove that, for any distributive lattice $L$, $L$ admits a Heyting implication if, and only if, $X_L$ is an Esakia space.\\


%f. Explain why, in an Esakia space, ${\uparrow} K$ is clopen for any clopen subset $K$.\\ 

%Let $L$ be a distributive lattice with dual Priestley space $X_L$, and suppose further that $X_L$ is an Esakia space.\\

%g. Show that, for any $a, b \in L$, the set $({\uparrow}[\widehat{a} \cap (\widehat{b})^c])^c$ is a clopen down-set.\\

%h. For any $a, b \in L$, let $a \Rightarrow b$ be the unique element such that $\widehat{a \Rightarrow b} = ({\uparrow}[\widehat{a} \cap (\widehat{b})^c])^c.$
%Prove that $\Rightarrow$ is a Heyting implication on $L$.\\

%Let $S \subseteq L$ be a finite subset, and denote by $M$ the (bounded) sublattice generated by $S$. Prove that, if $a \Rightarrow b$ 
%\newcommand{\IPL}{\mathrm{IPL}}
%{\bf Problem 3.} Let $P$ be a set of propositional variables, and let $\mathcal{T}(P)$ be the set of propositional terms (formulas) in variables $P$, built with the operations $\bot, \top, \vee, \wedge, \to$. For any subset $\Gamma \cup \{\phi\}$ of $\mathcal{T}(P)$, we say $\phi$ is an consequence  of $\Gamma$ in intuitionistic propositional logic (IPL), notation: $\Gamma \Vdash_{\IPL} \phi$, if, and only if, for every function $v \colon P \to H$ with $H$ a Heyting algebra, such that the unique homomorphism $V \colon \mathcal{T}(P) \to H$ extending $v$ satisfies $V(\gamma) = \top$ for all $\gamma \in \Gamma$, we have $V(\phi) = \top$. We write $\phi \equiv_{\IPL} \psi$ when $\{\phi\} \Vdash_{\IPL} \psi$ and $\{\psi\} \Vdash_{\IPL} \phi$.\\

%a. Explain why $\equiv_{\IPL}$ is a \emph{Heyting congruence}, i.e., an equivalence relation which respects all of the operations $\vee, \wedge, \to$ on $\mathcal{T}(P)$, such that the quotient $\mathcal{T}(P)/{\equiv_{\IPL}}$ is a Heyting algebra.

%{\bf Problem 3.} Give an example of a distributive lattice $L$ and an operator $f \colon L \times L \to L$ such that $f$ is not join-preserving.\\

{\bf Problem 3.} Let $P$ be a set of propositional variables, and let $\mathcal{T}(P)$ be the set of modal formulas in variables $P$, built with the operations $\bot, \top, \vee, \wedge, \to, \Diamond$.\\

a. Prove that, for any modal algebra $(B,\Diamond)$ and any function $v \colon P \to B$, there is a unique modal homomorphism $V \colon \mathcal{T}(P) \to B$ extending $v$.\\

%For $\phi, \psi \in \mathcal{T}(P)$, we say that the equation $\phi \approx \psi$ is \emph{valid} in $(B,\Diamond)$ if, for every function $v \colon P \to B$, we have $V(\phi) = V(\psi)$.

Let $\Lambda = \mathbf{K}\Sigma$ be the normal modal logic axiomatized by a set of modal formulas $\Sigma$. We say that $(B, \Diamond)$ is a \emph{$\Lambda$-algebra} if $(B,\Diamond)$ validates the equation $\sigma \approx \top$ for every formula $\sigma \in \Sigma$. (Note that, when $\Sigma = \emptyset$, a $\mathbf{K}$-algebra is just a modal algebra.) For $\phi, \psi \in \mathcal{T}(P)$, we write $\phi \equiv_\Lambda \psi$ if, and only if, the equation $\phi \approx \psi$ is valid in every $\Lambda$-algebra $(B,\Diamond)$.\\

b. Prove that $\equiv_{\Lambda}$ is a \emph{modal congruence}, i.e., $\equiv_{\Lambda}$ is an equivalence relation, and whenever $\phi_1 \equiv_{\Lambda} \psi_1$ and $\phi_2 \equiv_{\Lambda} \psi_2$, we have $\phi_1 \vee \phi_2 \equiv_{\Lambda} \psi_1 \vee \psi_2$, $\neg\phi_1 \equiv_{\Lambda} \neg\psi_1$, and $\Diamond\phi_1 \equiv_{\Lambda} \Diamond\psi_1$.\\

The \emph{Lindenbaum algebra for $\Lambda$} or \emph{free $\Lambda$-algebra} is defined as the quotient $F_{\Lambda}(P) := \mathcal{T}(P)/{\equiv_\Lambda}$.\\

c. Show that $F_{\Lambda}(P)$ is a $\Lambda$-algebra.\\

d. Show that, for any $\Lambda$-algebra $(B,\Diamond)$ and function $v \colon P \to B$, there exists a unique modal homomorphism $V \colon F_{\Lambda}(P) \to B$.\\

Let $(X,\tau,R)$ be a Boolean frame with dual modal algebra $(B,\Diamond)$, and let $v \colon P \to B$ be an admissible valuation.\\ %Write $\Diamond_R \colon $ for the operator defined by $\Diamond_R(U) := R^{-1}[U]$ for any $U \in \mathcal{P}(X)$. \\

e. Prove that the function $F_{\mathbf{K}}(P) \to B$, defined by $\phi \mapsto \{x \in X \ | \ (X,R,v), x \models \phi\}$, coincides with the modal homomorphism from (d).

%Let $\Sigma$ be a set of modal formulas.

%For any subset $\Gamma \cup \{\phi\}$ of $\mathcal{T}(P)$, we say $\phi$ is an consequence  of $\Gamma$ in modal logic $\Lambda$, notation: $\Gamma \Vdash_{\IPL} \phi$, if, and only if, for every function $v \colon P \to H$ with $H$ a Heyting algebra, such that the unique homomorphism $V \colon \mathcal{T}(P) \to H$ extending $v$ satisfies $V(\gamma) = \top$ for all $\gamma \in \Gamma$, we have $V(\phi) = \top$. We write $\phi \equiv_{\IPL} \psi$ when $\{\phi\} \Vdash_{\IPL} \psi$ and $\{\psi\} \Vdash_{\IPL} \phi$.

\newpage

{\bf Problem 4.} This problem asks you to fill in some details in the Rasiowa-Sikorski proof of completeness of predicate logic via the Stone dual space.

\begin{enumerate}
\item Let $L$ be a distributive lattice and $S \subseteq L$. Prove that, if $\widehat{\bigvee S} = \bigcup_{s \in S} \widehat{s}$, then there is a finite subset $F \subseteq S$ such that $\bigvee F = \bigvee S$.

\item Let $X$ be a topological space. The \emph{closure} of a subset $S$ of $X$ is defined as $\overline{S} := \bigcap \{ C \text{ closed }  \ | \ S \subseteq C\}$, i.e., $\overline{S}$ is the smallest closed set containing $S$. The \emph{interior} is defined as $\mathrm{int}(S) :=\bigcup \{ U \text{ open } \ | \ U \subseteq S\}$, i.e., the smallest open set contained in $S$. The \emph{boundary} of an open set $U$ is defined as $\partial U := \overline{U} \setminus U$. Prove that, for any open set $U \subseteq X$, $\partial U$ is closed, and that the interior of $\partial U$ is empty.

\item A topological space is called \emph{Baire} if any countable union of closed sets with empty interior has empty interior. Prove, or find a proof in the literature, that every compact Hausdorff space is Baire.

\end{enumerate}
\end{document}